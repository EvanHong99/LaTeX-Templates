\pagenumbering{arabic}
\section{问题重述和本组实现}

% TODO:在此基础上修改就行
本次作业需要利用不同算法来对用户打分进行估计,我们组在合理利用\verb|ItemAttribute.txt|的基础上,分别利用不同算法进行求解。
\begin{table}[ht!]
	\centering
	\begin{tabular}{@{}cccc@{}}
		\toprule
		\textbf{序号} & \textbf{项目主要特点}                    & \textbf{是否作业要求} & \textbf{本组完成情况}    \\
		\midrule
		1             & 实现基于协同过滤的预测算法               & √                     & √                        \\
		2             & 实现基于SVD的预测算法                    & √                     & √                        \\
		3             & 算法评价与分析                                 & √                     & √                        \\
		4             & \tabincell{c}{利用\verb|ItemAttribute.txt \\优化打分预测}         &                  & {\color[HTML]{CB0000} √}                        \\
		5             & 存储模长以及相关性矩阵,加速推理                 &                       & {\color[HTML]{CB0000} √} \\
		6             & 实现使用全局平均信息的预测算法 &                       & {\color[HTML]{CB0000} √} \\
		% 7             & 动态调整学习率 =========todo=========                          &                       & {\color[HTML]{CB0000} √} \\
		\bottomrule
	\end{tabular}
	\caption{作业要求及本组完成情况}
\end{table}
按照作业要求,本次结果我们汇报:
\begin{enumerate}
	\item 源代码、报告
	\item CF算法的参数文件,存放于data目录下
	\item 格式化好的结果文件,分别存放于\verb|./result_RSVD.txt|以及\verb|./result_CF.txt|
	\item 可执行文件
\end{enumerate}
\section{项目核心框架介绍}
本次项目核心代码为如下两个文件\verb|CF.py|、\verb|RSVD.py|。
下面介绍文件结构:
\begin{itemize}
	\item \verb|CF.py|:
		\iitem \texttt{calculate\_data\_bias}: 计算全局bias信息。
		\iitem \texttt{global\_model}: 以不同评分均值预测,包括总体均值,用户评分均值,item评分均值及综合均值。
		\iitem \texttt{collaborative\_filtering\_bias}: 协同过滤算法主函数。
		\iitem \texttt{predict}: 预测算法。
		\iitem \texttt{exec}: 封装好的主入口。
	\item \verb|RSVD.py|:
		\iitem \texttt{build\_lr}: 构造动态下降的学习率list。
		\iitem \texttt{gradient\_descent}: 梯度下降算法,更新矩阵参数。
		\iitem \texttt{cal\_rmse}: 计算rmse。
	\item \verb|./dataset/|: RSVD算法所需数据以及结果保存目录。
	\item \verb|./data/|: CF算法所需数据以及结果保存目录。
	\item \verb|result_RSVD.txt|: RSVD算法生成的格式化好的结果文件。(改文件不会更新,更新后的结果存储在相应目录下对应文件,若需要请阅读下文)
	\item \verb|result_CF.txt|: CF算法生成的格式化好的结果文件。(改文件不会更新,更新后的结果存储在相应目录下对应文件,若需要请阅读下文)
	\item \verb|rsvd.exe|: RSVD算法可执行文件。
	\item \verb|cf.exe|: CF算法可执行文件,直接利用参数进行推理。
\end{itemize}

\section{数据集概览}
\begin{minted}[
	mathescape,
	breaklines,
	% linenos,
	numbersep=4pt,
	gobble=2,
	frame=lines,
	framesep=2mm,
	obeytabs=true,
	tabsize=2
	]
	{text}
		train set
		用户数量: 19835
		商品数量: 455705
		评分数量: 5001507

		% 为了能够预测模型的优劣,我们将train set按照9:1切分为train data和validate data。
		% 并且为了预测的鲁棒性,我们保证了train data中包含了所有的品种,使得我们的训练结果不会对某种商品是茫然无知的。

		validate data
		用户数量: 455124
		商品数量: 118599
		评分数量: 455124

		train data
		用户数量: 4546383
		商品数量: 455705
		评分数量: 4546383

		test data
		用户数量: 19835
		商品数量: 28292
		评分数量: 119010

		item attribute
		商品数量: 624961
		商品属性统计信息:
		count    624961.000000
		mean     234039.891956
		std      207878.812544
		min           0.000000
		25%           0.000000
		50%      206339.000000
		75%      418894.000000
		max      624943.000000
		Name: attribute1, dtype: float64
		count    624961.000000
		mean     221134.948469
		std      207181.520174
		min           0.000000
		25%           0.000000
		50%      188235.000000
		75%      401338.000000
		max      624951.000000
		Name: attribute2, dtype: float64
\end{minted}

\section{项目配置方法}
\subsection{运行exe}

您可以直接运行根目录下的两个exe程序,无需变动项目结构。

输出结果分别存放于\verb|./dataset/test1.txt|以及\verb|./data/result_CF.txt|(分别对应RSVD和CF)

\subsection{使用Python3解释器运行源代码}
本部分供教授和助教检查所用。
\begin{enumerate}
	\item 首先,请您保证两个python文件、exe文件与data、dataset文件夹处于同一目录下。您可以用您本地的python解释器运行,也可以直接运行我们打包好的exe程序。
	\item 运行\verb|RSVD.py|:
		\iitem 对于\texttt{RSVD} 算法,您可以在main函数开头调整您所需要的参数,我们已经预设了一些值。
		\iitem 您可以运行RSVD的main函数,并在训练结束时进行打分预测,结果存放于\texttt{./dataset/test1.txt}
	\item 运行\verb|CF.py|:
		\iitem 程序入口已经封装在了\texttt{exec}函数中,您可以在\texttt{CF.py}的\texttt{main}函数中进行路径调整。
		\iitem 您可以在\texttt{exec}函数中选择运行不同的环节,包括训练全局平均模型、训练协同过滤模型、加载预训练参数并直接预测。
		\iitem 预测结果存放于\texttt{data/result\_CF.txt}中。
\end{enumerate}

\section{实验思路及实践路径}
% 2.	Details of the algorithms; 
% 3.	Experimental results of the recommendation algorithms (RMSE, training time, space consumption); 
% 4.	Theoretical analysis or/and experimental analysis of the algorithms. 
% \subsection{实验思路}

\subsection{数据预处理}

\subsubsection{读取数据}
根据不同的算法的需求,我们对于读取数据的策略有细节上的不同(并且命名方式有些许差异,但是阅读源码时您一定能理解,因为命名都很直接),但是大致思路如下:
\begin{breakablealgorithm} 
	\caption{读取数据及预处理} 
	\begin{algorithmic}[1] %每行显示行号  
		\Require filepath
		\Ensure dataframe(RSVD算法) or dict(CF算法)
		\Function {load\_data}{$self,filepath$} 
			\State 打开文件
			\While{\texttt{readline}不为空} 
				\If	{识别到了"|"}
					\State 记录用户id以及打分个数
				\Else 
					\State 如果是nan,那么我们特殊处理
					\State 将下面的评分结果以及用户id以不同的方式存储起来(dataframe或者dict)
				\EndIf
			\EndWhile
		\EndFunction
	\end{algorithmic}  
\end{breakablealgorithm}

\subsubsection{写出格式化结果}

实现思路较为简单,这里不再赘述。

关键点是一些特殊值的处理,如防止结果出现小于0大于100等少数情况。


\subsubsection{划分训练集以及验证集}

按照 divide\_size 定义的比例划分成train与validate数据集,同时保证划分后train中包含所有item

原因是后续需要计算item与item之间相似度,如果train中不存在则无法计算,影响效果。


\subsubsection{预处理\texttt{ItemAttribute}}}

我们在实现的时候,也是先将数据存储在dataframe中,方便对一些特殊值做处理。

之后为了查询效率(dataframe查询检索真的很慢很慢),我们将其转为dict存储,利用底层的hash来使得检索复杂度降低到O(1)。

此外,比较重要的一点是,我们为了加速预测,将很多中间结果存储在了硬盘中。
在预处理\texttt{ItemAttribute}的过程中,我们为了方便地计算item之间的相似度,
将每个物品的模长提前计算好并且存储到磁盘中,方便下次直接使用。

代码如下:
\begin{minted}[
	mathescape,
	breaklines,
	% linenos,
	numbersep=4pt,
	gobble=2,
	frame=lines,
	framesep=2mm,
	obeytabs=true,
	tabsize=2
	]
	{python}
		self.item_attributes["norm"] = 
			(self.item_attributes["attribute1"] ** 2 
			+ self.item_attributes["attribute2"] ** 2) ** (1 / 2)
\end{minted}


\subsubsection{计算bias}

为了引入用户的打分偏好以及物品的受欢迎程度,我们需要计算他们与平均评分之间的差值,以对预测结果进行修正。
\begin{breakablealgorithm} 
	\caption{计算bias} 
	\begin{algorithmic}[1] %每行显示行号  
		\Require train\_data
		\Ensure 全局平均值,dict\{用户id:打分偏好偏移\},dict\{物品id:打分偏好偏移\}
		\Function {calculate\_data\_bias}{$self$} 
			\State 求全局平均
			\State 求每个用户平均打分与全局平均的差值,存入dict
			\State 求每个物品平均打分与全局平均的差值,存入dict
		\EndFunction
	\end{algorithmic}  
\end{breakablealgorithm}

\subsection{方法1:全局模型+bias}

该算法思路较为简单,就是利用之前计算好的各个均值、偏移信息来构建预测结果。

而实现这个算法的出发点也是很朴素的,就是寻找最简单的规律(平均值、偏移平均值)来对未知进行预测。
\begin{enumerate}
	\item \texttt{Global average:} 所有评分的均值作为预测结果,即仅有一个预测值。RMSE:  38.13540922065373
	\item \texttt{User average:} 所有评分的均值作为预测结果+用户偏移。RMSE:  29.552371104008376
	\item \texttt{Item average:} 所有评分的均值作为预测结果+物品偏移。RMSE:  35.82071948493244
	\item \texttt{Global effects:} 所有评分的均值作为预测结果+用户偏移+物品偏移。RMSE:  30.946452267391507
\end{enumerate}

Average running time: 2.4483330249786377 Seconds

\subsection{方法2:协同过滤+bias}
% 2.	Details of the algorithms; 
% 3.	Experimental results of the recommendation algorithms (RMSE, training time, space consumption); 
% 4.	Theoretical analysis or/and experimental analysis of the algorithms. 
% RMSE:  22.65362302506816
该算法需要利用到之间计算好的模长数据。

此外,比较重要的一点是,我们会在程序运行中间阶段计算物品间的相似度并存储在磁盘中,
如果下次再次碰到计算这两个物品相似度的情况,我们可以快速读取之前计算好的结果。
并且由于相似度是相互的,因此为了节省磁盘开销,我们只需要存储对称矩阵的一半即可。

\textbf{[注]}:由于我们在测试的时候是使用自己切割的validation集而非\texttt{test.txt},因此为了方便地处理两种情况,
项目中包含了两个函数\texttt{collaborative\_filtering\_bias}以及\texttt{predict()}。
但是实现原理是相同的,简化起见以下只介绍
\texttt{collaborative\_filtering\_bias}。
您在测试的时候将会使用到predict函数来和真实的测试集进行比较。
\begin{breakablealgorithm} 
	\caption{协同过滤+bias} 
	\begin{algorithmic}[1] %每行显示行号  
		\Require 我们切分的train\_data以及validate\_data,之前计算的bias、相似度中间结果
		\Ensure RMSE
		\Function {collaborative\_filtering\_bias}{$self$} 
			\For{遍历数据集}
				\State 计算偏置b\_x,包含全局平均信息以及物品偏好偏移,用于修正物品相关性。
				\For {与该物品相关的其他物品}
					\If {已经计算过相似度}
						break
					\Else	\State 进行相似度计算:
							\State 属性相似度为cosine相似度。
							\State pearson相似度用到pearson相关系数,并且需要保证用户评分数高于20避免出现较大偏差。
					\EndIf
				\EndFor
				\State 取最相似的100个物品(If Any)
				\For{遍历所有相似物品以求取加权平均}
					\State 更新偏置,加入用户偏好偏移。
					\State 加入到平均值中。
				\EndFor
			\EndFor
			\State 计算RMSE误差。
			\State \Return RMSE
		\EndFunction
	\end{algorithmic}
\end{breakablealgorithm}

实现代码:
\begin{minted}[
	mathescape,
	breaklines,
	% linenos,
	numbersep=4pt,
	gobble=2,
	frame=lines,
	framesep=2mm,
	obeytabs=true,
	tabsize=2
	]
	{python}
		def collaborative_filtering_bias(self):
			# 协同过滤算法
			predict_rate = []
			for index, row in enumerate(self.train_test_data.values):
				user, item_x, pred_score_x = row
				score_x = 0
				b_x = self.bias["deviation_of_item"][item_x] + self.bias["overall_mean"]
				similar_item = {}
				# 计算物品相似度
				for item_y in self.user_item_train_data[user].keys():
					if item_x in self.similarity_map and item_y in self.similarity_map[item_x]:
						similar_item[item_y] = self.similarity_map[item_x][item_y]
					elif item_y in self.similarity_map and item_x in self.similarity_map[item_y]:
						similar_item[item_y] = self.similarity_map[item_y][item_x]
					else:
						b_y = self.bias["deviation_of_item"][item_y] + self.bias["overall_mean"]
						if self.item_attributes[item_x][2] == 0 or self.item_attributes[item_y][2] == 0:
							attribute_similarity = 0
						else:
							attribute_similarity = (self.item_attributes[item_x][0] * self.item_attributes[item_y][0]
													+ self.item_attributes[item_x][1] * self.item_attributes[item_y][1]) \
												/ (self.item_attributes[item_x][2] * self.item_attributes[item_y][2])
						norm_x = 0
						norm_y = 0
						pearson_similarity = 0
						count = 0
						for same_user, score in self.item_user_train_data[item_x].items():
							if same_user not in self.item_user_train_data[item_y]:
								continue
							count += 1
							pearson_similarity += (self.item_user_train_data[item_x][same_user] - b_x) * (
									self.item_user_train_data[item_y][same_user] - b_y)
							norm_x += (self.item_user_train_data[item_x][same_user] - b_x) ** 2
							norm_y += (self.item_user_train_data[item_y][same_user] - b_y) ** 2
						if count < 20:
							pearson_similarity = 0
						if pearson_similarity != 0:
							pearson_similarity /= (norm_x * norm_y) ** (1 / 2)

						similarity = (pearson_similarity + attribute_similarity) / 2

						if item_x not in self.similarity_map:
							self.similarity_map[item_x] = {}
						self.similarity_map[item_x][item_y] = similarity

						similar_item[item_y] = similarity

				similar_item = sorted(similar_item.items(), key=lambda item: item[1], reverse=True)
				b_x = self.bias['overall_mean'] + self.bias['deviation_of_item'][item_x] + self.bias['deviation_of_user'][
					user]
				norm = 0
				for i, (item_y, similarity) in enumerate(similar_item):
					if i > 100:
						break
					b_y = self.bias['overall_mean'] + self.bias['deviation_of_item'][item_y] + \
						self.bias['deviation_of_user'][user]
					score_x += similarity * (self.item_user_train_data[item_y][user] - b_y)
					norm += similarity
				if norm == 0:
					score_x = 0
				else:
					score_x /= norm
				score_x += b_x
				score_x = score_x if score_x > 0 else 0
				score_x = score_x if score_x < 100 else 100
				predict_rate.append(score_x)
				if index % 500 == 0 and index != 0:
					print("已预测", index)
				if index % 5000 == 0 and index != 0:
					print("RMSE: ", cal_RMSE(predict_rate, self.train_test_data['score'][:index + 1]))
				if index % 200000 == 0 and index != 0:
					with open("data/similarity_map.pickle", 'wb') as f:
						pickle.dump(self.similarity_map, f)
			print("RMSE: ", cal_RMSE(predict_rate, self.train_test_data['score']))

\end{minted}

\subsection{方法3:RSVD及矩阵初始化策略}

\subsubsection{初始化策略}
%已完成,无需改动
为了保证初始化结果能够具有一个可以接受的上界,首先考虑打分一定介于$[0,100]$,因此若对于$\forall u,i$,我们都假设初始化为$\hat{A}_{ui}^{(0)}=50$
,则一定有:
$$RMSE=\sqrt{\frac{\sum_{dataset} (A-50)}{n}}<=50$$
所以我们初始化时已然具有上界50,不会比它更差了。此时,假设$P,Q$矩阵每个元素都相等且为$a$。


$\forall u,i$则:
$$\hat{A}_{ui}=P[u,:]Q[:,i]=fa^2=50$$
求解得:
$$a=\sqrt{\frac{50}{f}}$$
因此我们的代码为:
\begin{minted}[
	mathescape,
	breaklines,
	% linenos,
	numbersep=4pt,
	gobble=2,
	frame=lines,
	framesep=2mm,
	obeytabs=true,
	tabsize=2
	]
	{python}
		U = max(max(train_df['user']), max(validation_df['user'])) + 1
		I = max(max(train_df['item']), max(validation_df['item'])) + 1
		P = np.ones(U, factor)* (np.sqrt(np.mean(train_df['score']) / factor))
		Q = np.ones(factor, I)* (np.sqrt(np.mean(train_df['score']) / factor))
\end{minted}

在验证集中,初始化时就能达到RMSE=38的效果。


\subsubsection{建模过程}
%已完成,无需改动
我们定义用户数量为$U$,商品数量为$I$,factor定义为$F$。我们定义预测矩阵:
$$\hat{A}=P_{U\times F}\times Q_{F\times I}$$


对于评价指标,我们定义:
$$RMSE=\sqrt{\frac{\sum_{dataset} (A-\hat{A})^2}{n}}$$
对于目标函数,我们定义为:
$$L = \sum_{dataset} (A-\hat{A})^2+\lambda_p \Vert P[u,:]\Vert + \lambda_q  \Vert Q[:,i]\Vert$$
根据梯度下降法,我们操作:
$$Q[:,i]:=Q[:,i]-\eta \nabla Q[:,i] $$
$$P[u, :]:=P[u, :]-\eta \nabla P[u, :] $$
根据矩阵求导法则,可以进一步推得:
$$Q[:, i] :=\eta( e_{ui}P[u, :] - \lambda_q Q[:, i]) $$
$$P[u, :] :=\eta( e_{ui}Q[:, i] - \lambda_q P[u, :]) $$
其中$$e_{ui}=A_{ui}-\hat{A}_{ui}$$
$$\hat{A}_{ui}=P[u, :]\cdot Q[:, i]$$

以下是具体实现的核心代码:
\begin{minted}[
	mathescape,
	breaklines,
	% linenos,
	numbersep=4pt,
	gobble=2,
	frame=lines,
	framesep=2mm,
	obeytabs=true,
	tabsize=2
	]
	{python}
		def gradient_descent(P, Q, search_df, lambda_p, lambda_q, lr=0.01):
			...
			for _, line in enumerate(search_df.values):  # enumerate(search_df.iterrows()):
				sys.stdout.write('\r梯度下降更新进度-->' + str(round(_ / len(search_df), 4)))
				sys.stdout.flush()
				u = line[0]
				i = line[1]
				s = line[2]
				eui = s - np.matmul(P[u, :], Q[:, i])

				Q[:, i] += lr * eui * P[u, :].T - lambda_q * Q[:, i] * lr
				P[u, :] += lr * eui * Q[:, i].T - lambda_p * P[u, :] * lr
			...

		def cal_rmse(ds, P, Q):
			...
			true_val = np.array(ds['score'])
			pred_val = []
			for _i, line in enumerate(ds.values):
				u = line[0]
				i = line[1]
				pred_val.append(np.matmul(P[u, :], Q[:, i]))
				
			rmse = (np.sum((np.array(pred_val) - true_val) ** 2) / len(true_val)) ** 0.5
			...
\end{minted}

\section{不同算法结果对比}

% Please add the following required packages to your document preamble:
% \usepackage{graphicx}
\begin{table}[!htbp]
	\centering
	\begin{tabular}{cccc}
	\toprule
	\multicolumn{1}{c}{算法名称} & \multicolumn{1}{c}{训练时间} & \multicolumn{1}{c}{预测时间} & \multicolumn{1}{c}{RMSE} \\ \midrule
	Global average & ~ & 2.44s & 38.1 \\ 
	User average & ~ & 2.44s & 29.5 \\
	Item average & ~ & 2.44s & 35.8 \\
	Global effects & ~ & 2.44s & 30.9 \\
	CF & 10h & 342.4s & 24.6 \\
	RSVD & ~ & 约36min & 30.2 \\
	\bottomrule
	\end{tabular}%
	\caption{不同算法对比}
	\label{tab:tab1}
\end{table}
[注]:除了CF算法,其余算法均需要先训练再预测,时间之和填写在预测时间一栏。
此外,由于前四种算法较为相似,并且本实验重心并不于此,因此我们仅取平均时长作为其时长(实际上每个算法时长也确实差不离)。

可以看到,利用带bias矫正的CF算法在我们的训练集和验证集上表现较为良好,并且训练好后推理速度较快。

但是不可忽视的是基于用户平均的算法,简单朴素的思想以及快速的训练推理仍旧能带来较为不错的结果,着实出乎我们的意料。

以下是两种算法预测结果的部分展示,具体结果请查看\texttt{result\_CF.txt}以及\texttt{result\_RSVD.txt}。
\begin{minted}[
	mathescape,
	breaklines,
	% linenos,
	numbersep=4pt,
	gobble=2,
	frame=lines,
	framesep=2mm,
	obeytabs=true,
	tabsize=2
	]
	{text}
		% CF
		0|6
		208031 78
		193714 38
		393064 79
		207030 91
		112040 6
		464229 68
		1|6
		55971 82
		583090 74
		180171 93
		617646 89
		175835 95
		553890 89
		2|6
		48916 76
		238557 33
		61148 0
		378073 32
		17863 41
		187943 24


		% RSVD
		0|6
		208031 78
		193714 69
		393064 85
		207030 80
		112040 69
		464229 91
		1|6
		55971 79
		583090 81
		180171 87
		617646 88
		175835 93
		553890 87
		2|6
		48916 58
		238557 57
		61148 28
		378073 67
		17863 56
		187943 50
\end{minted}

\section{遇到的问题及解决方法}

\begin{itemize}
	\item 利用dataframe进行索引带来的效率低下,后来改用O(1)的dict解决了这个问题,大大加速了训练。
	\item 在SVD分解中,常常出现由于不同factor值的设置导致的overflow问题,进而导致RMSE为nan。为此我们选择了较为满意的参数组合来避免这个问题,但是带来的结果就是无法进行grid search。
\end{itemize}

\section{总结}

本次作业中,小组成员均在代码、实验和报告中有所贡献。

我们利用了不同的算法,从简单到复杂,较为全面地了解了推荐系统的部分主流实现方式。
小组成员的科研和工程能力在本次作业中得到了很好地锻炼。
同时,存储预处理好的参数来加速程序也是从中获得的处理大数据问题时的经验。