\input{regression-test.tex}
\documentclass[degree=doctor]{thuthesis}

\begin{document}
\START


\frontmatter
\begin{abstract}
\end{abstract}

\clearpage
\setcounter{page}{2}
\begin{abstract*}
\end{abstract*}

\clearpage
\setcounter{page}{4}
\showoutput
\tableofcontents
\clearpage
\OMIT

\clearpage
\setcounter{page}{8}
\listoffiguresandtables

\clearpage
\setcounter{page}{9}
\begin{denotation}
  \item
\end{denotation}



\mainmatter
\chapter{引言}
\section{研究背景}
\subsection{城市排水系统的复杂性}

\clearpage
\setcounter{page}{4}
\subsection{城市精细化管理对城市排水系统设计提出新要求}

\clearpage
\setcounter{page}{6}
\subsection{传统规划方法的局限与不足}

\clearpage
\setcounter{page}{7}
\section{问题的提出}

\clearpage
\setcounter{page}{8}
\section{研究目的、内容与意义}
\subsection{研究目的}
\subsection{研究内容}
\subsection{研究意义}

\clearpage
\setcounter{page}{9}
\section{论文结构}


\clearpage
\setcounter{page}{11}
\chapter{文献综述}
\section{城市排水系统规划设计内容的研究进展综述}
\subsection{“网”的简单构建}

\clearpage
\setcounter{page}{12}
\subsection{“网+厂”的传统规划}

\clearpage
\setcounter{page}{15}
\subsection{“面+网+厂”的空间设计}

\clearpage
\setcounter{page}{18}
\section{城市排水系统设计方法和模型的应用综述}
\subsection{“网+厂”的传统规划方法和模型}

\clearpage
\setcounter{page}{23}
\subsection{“面+网+厂”的空间设计方法和模型}

\clearpage
\setcounter{page}{27}
\section{城市排水系统设计中考虑不确定性的实践综述}
\subsection{影响城市排水系统设计的不确定性因素及其分类}

\clearpage
\setcounter{page}{28}
\subsection{考虑不确定性因素的城市排水系统设计实践}

\clearpage
\setcounter{page}{38}
\section{本章小结}


\clearpage
\setcounter{page}{40}
\chapter{不确定条件下分流制城市排水系统优化设计方法研究}
\section{方法框架}

\clearpage
\setcounter{page}{41}
\section{问题识别}

\clearpage
\setcounter{page}{46}
\section{数据收集和不确定条件量化预测}
\subsection{数据收集生成基准设计条件}

\clearpage
\setcounter{page}{47}
\subsection{不确定条件量化预测生成评估情景集合}

\clearpage
\setcounter{page}{53}
\section{基准条件下多目标优化}
\subsection{雨水系统设计}

\clearpage
\setcounter{page}{57}
\subsection{污水系统设计}

\clearpage
\setcounter{page}{59}
\section{不确定条件下系统性能评估}

\clearpage
\setcounter{page}{60}
\subsection{雨水系统性能评估}

\clearpage
\setcounter{page}{61}
\subsection{污水系统性能评估}

\clearpage
\setcounter{page}{62}
\section{方案优选与最优集合推荐}

\clearpage
\setcounter{page}{63}
\section{本章小结}


\clearpage
\setcounter{page}{64}
\chapter{含有不确定性参数的城市排水系统优化设计模型}
\section{整体框架}

\clearpage
\setcounter{page}{65}
\section{基本假设}
\subsection{城市地块与设计单元}

\clearpage
\setcounter{page}{67}
\subsection{设施空间位置的选择原则}

\clearpage
\setcounter{page}{68}
\subsection{径流调蓄设施的能力及其空间分布}

\clearpage
\setcounter{page}{69}
\subsection{其他基本假设}
\section{模型输入}
\subsection{设计单元属性数据}

\clearpage
\setcounter{page}{70}
\subsection{排水和再生水数据}

\clearpage
\setcounter{page}{72}
\subsection{空间信息数据}

\clearpage
\setcounter{page}{73}
\subsection{设施参考数据}

\clearpage
\setcounter{page}{74}
\subsection{成本属性数据}

\clearpage
\setcounter{page}{77}
\subsection{模拟降雨数据}
\subsection{产流拟合参数数据}

\clearpage
\setcounter{page}{78}
\subsection{雨水系统和污水系统输入数据一览表}

\clearpage
\setcounter{page}{79}
\section{模型构建}
\subsection{雨水系统模型构建}

\clearpage
\setcounter{page}{85}
\subsection{污水系统模型构建}

\clearpage
\setcounter{page}{91}
\section{算法设计}
\subsection{整体设计思路}
\subsection{雨水系统算法设计}

\clearpage
\setcounter{page}{99}
\subsection{污水系统算法设计}

\clearpage
\setcounter{page}{105}
\subsection{算法性能}

\clearpage
\setcounter{page}{107}
\section{模型输出}

\clearpage
\setcounter{page}{108}
\section{本章小结}


\clearpage
\setcounter{page}{109}
\chapter{案例研究:昆明市城北片区排水系统设计}
\section{昆明市城北片区概况}

\clearpage
\setcounter{page}{110}
\section{研究区域概化及数据收集}
\subsection{研究区域概化}

\clearpage
\setcounter{page}{111}
\subsection{基础数据收集和不确定条件量化预测}

\clearpage
\setcounter{page}{115}
\section{设计输入信息}
\subsection{基准设计条件}

\clearpage
\setcounter{page}{120}
\subsection{不确定性评估情景集合}
\section{基准条件下的系统设计输出}
\subsection{雨水系统设计方案}

\clearpage
\setcounter{page}{124}
\subsection{污水系统设计方案}

\clearpage
\setcounter{page}{128}
\section{不确定条件下的系统性能评估}
\subsection{雨水系统性能评估}

\clearpage
\setcounter{page}{130}
\subsection{污水系统性能评估}

\clearpage
\setcounter{page}{134}
\section{推荐方案集合}
\subsection{雨水系统推荐方案}

\clearpage
\setcounter{page}{135}
\subsection{污水系统推荐方案}

\clearpage
\setcounter{page}{137}
\section{与城北片区现状系统的性能比较}

\clearpage
\setcounter{page}{139}
\section{本章小结}


\clearpage
\setcounter{page}{141}
\chapter{不确定条件下城市排水系统设计规律识别与分析}
\section{雨水系统}
\subsection{系统分散/集中程度对雨水系统性能的影响}

\clearpage
\setcounter{page}{143}
\subsection{管道设计重现期与径流空间调蓄的协调关系}

\clearpage
\setcounter{page}{148}
\section{污水系统}
\subsection{设计输入条件变化对污水系统最优解集的影响}

\clearpage
\setcounter{page}{151}
\subsection{DWF 设计原则对污水系统性能评估结果的影响}

\clearpage
\setcounter{page}{153}
\section{本章小结}


\clearpage
\setcounter{page}{155}
\chapter{结论与建议}
\section{结论}

\clearpage
\setcounter{page}{158}
\section{建议}



\backmatter
\clearpage
\setcounter{page}{159}
\begin{thebibliography}{0}
\end{thebibliography}


\clearpage
\appendix
\setcounter{page}{173}
\chapter{雨水系统和污水系统设计模型的程序文件一览表}


\clearpage
\setcounter{page}{179}
\chapter{空间降尺度和时间降尺度模型结构与主要程序一览表}


\clearpage
\setcounter{page}{184}
\begin{acknowledgements}
\end{acknowledgements}

\clearpage
\setcounter{page}{185}
\statement

\clearpage
\setcounter{page}{186}
\begin{resume}
\end{resume}

\clearpage
\setcounter{page}{188}
\chapter{指导教师学术评语}

\clearpage
\setcounter{page}{189}
\chapter{答辩委员会决议书}


\OMIT
\end{document}
