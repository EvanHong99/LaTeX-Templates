%!TEX program = xelatex
% 完整编译: xelatex -> bibtex -> xelatex -> xelatex
%-- coding: UTF-8 --
\documentclass{article}
\usepackage[UTF8]{ctex}
\usepackage{ctex}
\usepackage{graphicx}
\usepackage{color,framed}%文本框
\usepackage{listings}
\usepackage{amssymb}
\usepackage{enumerate}
\usepackage{xcolor}
\usepackage{bm} 
\usepackage{lastpage}%获得总页数
\usepackage{fancyhdr}
\usepackage{tabularx}  
\usepackage{geometry}
\usepackage{minted}
\usepackage{zhlipsum}
\usepackage[colorlinks = true]{hyperref}
\usepackage{array}
\newcommand{\ccr}[1]{\makecell{{\color{#1}\rule{1cm}{1cm}}}}

%书签

%页面边距
\geometry{a4paper,left=2.3cm,right=2.3cm,top=2.7cm,bottom=2.7cm}

%代码段设置
\lstset{numbers=left,
basicstyle=\tiny,
numberstyle=\tiny,
keywordstyle=\color{blue!70},
commentstyle=\color{red!50!green!50!blue!50},
frame=single, rulesepcolor=\color{red!20!green!20!blue!20},
escapeinside=``
}



\pagestyle{fancy}
\lhead{编译原理作业}
\rhead{\rightmark}
\cfoot{\thepage\ of\ \pageref{LastPage}}%当前页 of 总页数
\renewcommand{\headrulewidth}{0.4pt}%改为0pt即可去掉页眉下面的横线
\renewcommand{\footrulewidth}{0.4pt}%改为0pt即可去掉页脚上面的横线



\begin{document}
\renewcommand {\thefigure} {\thesection{}.\arabic{figure}}%图片按章标号
\renewcommand{\figurename}{图}
\renewcommand{\contentsname}{目录}           % 对Contents 进行汉化为目录


\title{\zihao{2}\kaishu{\textbf{编译原理作业1}}}
\author{\zihao{4}\ \kaishu{1811363 \\ 洪一帆}\ }
\date{September 2020}
\maketitle

\newpage
\tableofcontents

\newpage

\section{Section1}

\zhlipsum[1]
% \begin{lstlisting}
\begin{minted}[mathescape,
               linenos,
               numbersep=5pt,
               gobble=0,
               frame=lines,
               framesep=2mm,
               ]{c++}
    #include<stdio.h>
    int main(){
        int a,b;
        // 输入变量
        scanf("%d%d",&a,&b);
        // 输出结果
        printf("Hello World %d\n",a+b);
        return 0;
    }
\end{minted}

\subsection{预编译}
pdflatex -shell-escape 1811363\_洪一帆.tex
gcc main.c -E -o main.i

主要用于处理\#有关的代码

删除\#define,展开所有宏定义。例\#define portnumber 3333

处理条件预编译 \#if, \#ifdef, \#if, \#elif,\#endif

处理“\#include”预编译指令,将包含的“\.h”文件插入对应位置。这可是递归进行的,文件内可能包含其他“\.h”文件。

删除所有注释。/**/,//。

添加行号和文件标识符。用于显示调试信息:错误或警告的位置。

保留\#pragma编译器指令。(1)设定编译器状态,(2)指示编译器完成一些特定的动作。


\subsection{Subsection1}

特殊字符$\lambda$


\begin{quote}
    鲁迅:这是一段引用。
\end{quote}

\subsection{Subsection2}
\zhlipsum[1]

\subsection{Subsection2}

\paragraph{\texttt{段落}} \zhlipsum[1]

\begin{enumerate}
    \item 列表1 \footnote{这里是脚注。}。
    \item 列表2:带有\href{https://www.runoob.com/}{超链接}。
\end{enumerate}


\section{Section2}

\subsection{Subsection1: 另一种代码段}

\begin{lstlisting}
sudo apt update
sudo apt install build-essential
sudo apt install gcc-multilib
sudo apt install -y flex
sudo apt install -y bison
sudo apt install -y qemu
sudo apt install -y qemu-system
sudo apt install -y qemu-user
\end{lstlisting}

\subsection{Subsection2: 你可能要加引用}

看\href{https://www.overleaf.com/learn/latex/bibliography_management_with_bibtex}{这里}。

\end{document}


